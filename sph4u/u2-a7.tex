\documentclass{article}

\usepackage{fancyhdr}
\usepackage{extramarks}
\usepackage{amsmath}
\usepackage{amsthm}
\usepackage{amsfonts}
\usepackage{siunitx}
\usepackage{textcomp}
\usepackage{tikz}

\usetikzlibrary{automata,positioning}

\topmargin=-0.45in
\evensidemargin=0in
\oddsidemargin=0in
\textwidth=6.5in
\textheight=9.0in
\headsep=0.25in

\linespread{1.1}

\pagestyle{fancy}
\lhead{\hmwkAuthorName}
\chead{\hmwkClass}
\rhead{\firstxmark}
\lfoot{\lastxmark}
\cfoot{\thepage}

\renewcommand\headrulewidth{0.4pt}
\renewcommand\footrulewidth{0.4pt}

\setlength\parindent{0pt}

\newcommand{\enterProblemHeader}[1]{
    \nobreak\extramarks{}{Problem \arabic{#1} continued on next page\ldots}\nobreak{}
    \nobreak\extramarks{Problem \arabic{#1} (continued)}{Problem \arabic{#1} continued on next page\ldots}\nobreak{}
}

\newcommand{\exitProblemHeader}[1]{
    \nobreak\extramarks{Problem \arabic{#1} (continued)}{Problem \arabic{#1} continued on next page\ldots}\nobreak{}
    \stepcounter{#1}
    \nobreak\extramarks{Problem \arabic{#1}}{}\nobreak{}
}

\setcounter{secnumdepth}{0}
\newcounter{partCounter}
\newcounter{homeworkProblemCounter}
\setcounter{homeworkProblemCounter}{1}
\nobreak\extramarks{Problem \arabic{homeworkProblemCounter}}{}\nobreak{}

\newenvironment{homeworkProblem}[1][-1]{
    \ifnum#1>0
        \setcounter{homeworkProblemCounter}{#1}
    \fi
    \section{Problem \arabic{homeworkProblemCounter}}
    \setcounter{partCounter}{1}
    \enterProblemHeader{homeworkProblemCounter}
}{
    \exitProblemHeader{homeworkProblemCounter}
}

\newcommand{\hmwkTitle}{Asgnmt. 7}
\newcommand{\hmwkDueDate}{August 1, 2024}
\newcommand{\hmwkClass}{SPH4U}
\newcommand{\hmwkClassTime}{Unit 2}
\newcommand{\hmwkClassInstructor}{Naitrim Persaud}
\newcommand{\hmwkAuthorName}{\textbf{Gregory Olegovich Polstvin}}

\title{
    \vspace{2in}
    \textmd{\textbf{\hmwkClass:\ \hmwkTitle}}\\
    \normalsize\vspace{0.1in}\small{Due\ on\ \hmwkDueDate}\\
    \vspace{0.1in}\large{\text{\hmwkClassInstructor}}
    \vspace{3in}
}

\author{\hmwkAuthorName}
\date{}

\renewcommand{\part}[1]{\textbf{\large Part \Alph{partCounter}}\stepcounter{partCounter}\\}

\newcommand{\solution}{\textbf{\large Solution}}

\begin{document}

\maketitle

\pagebreak

\begin{homeworkProblem}
A ski-plane with a total mass of \SI{1200}{\kg} lands towards the west on a
frozen lake at \SI{30.0}{\meter\per\second}. The coefficient
of kinetic friction between the skis and the ice is 0.200.
How far does the plane slide before coming to a stop?
\\

\begin{solution}

\end{solution}

% TODO: Fill in missing information and make it more detailed.

\textbf{uhhhhhhhhhhh}
    \[
    \mu mg = ma
    \]

    \[
    a = \mu g
    \]

    \[
    a = 0.20 \cdot 9.81
    \]

    \[
      a = \SI{1.96}{m/s^2}
    \]
    \\
    \textbf{Distance Traveled}

    \[
      v^2 - u^2 = 2ad
    \]
    \[
      d = \frac{-u^2}{2a}
    \]
    \[
      d = \frac{-(30)^2}{2 \cdot -1.96}
    \]
    \[
      d = \SI{229.59}{m}
    \]
    \\

\end{homeworkProblem}

\pagebreak

\begin{homeworkProblem}

  Two Arctic explorers are pulling a supply sled with a total mass of
  160 kg. Alphonse pulls the sled with a force of 50.0 N [E 25° N],
  while Bela pulls the sled with a force of 80.0 N [E 20.0°
  S]. Determine the acceleration of the sled. \\

  \begin{solution}
    \\
    Acceleration of the sled is \SI{0.75}{m/s^2}
    \\
  \end{solution}

  \textbf{Net Force on Sled} \\
  We can find the acceleration
  acting on the sled using Newton's Second Law of Motion.
    \[
   \]
   \[
  \]
  \[
    
  \]
  \\
\end{homeworkProblem}

\begin{homeworkProblem}
A box with a mass of 10.0 kg is placed on a hill that includes upwards
at an angle of 30.0°. The coefficient of static friction between the
box and the hill is 0.582, which the coefficient of kinetic friction
is 0.528. If the box is given an initial push to start it moving down
the hill, will it continue moving down the hill when the initial force
is removed, or will it coast to a stop? Explain.

  \[
  \]
  
  \[
  \]

  \[
  \]
  \\
  
\end{homeworkProblem}

\begin{homeworkProblem}

  A mass m1 of 250 g is on a table connected to a massless pulley, as
  shown. The coefficient of friction between m1 and the table is 0.228.
  What is the maximum value of m2 before m1 starts sliding across the table?

\end{homeworkProblem}

\begin{homeworkProblem}
  Sandor fills a bucket with water and whirls it in a vertical circle to
  demonstrate that the water will not spill from the bucket at the top
  of the loop. If the length of the rope from his hand to the centre
  of the bucket is 1.24 m, what is the minimum tension in the rope
  (at the top of the swing)? How slow can he swing the bucket?
  Explain your answer.
  
\end{homeworkProblem}

\end{document}
