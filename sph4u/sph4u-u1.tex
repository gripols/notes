\documentclass[12pt, oneside]{article} 
\usepackage{amsmath, amsthm, amssymb, calrsfs, wasysym, verbatim, bbm, color, graphics, geometry}

\geometry{tmargin=.75in, bmargin=.75in, lmargin=.75in, rmargin = .75in}  

\newcommand{\R}{\mathbb{R}}
\newcommand{\C}{\mathbb{C}}
\newcommand{\Z}{\mathbb{Z}}
\newcommand{\N}{\mathbb{N}}
\newcommand{\Q}{\mathbb{Q}}
\newcommand{\Cdot}{\boldsymbol{\cdot}}

\newtheorem{thm}{Theorem}
\newtheorem{defn}{Definition}
\newtheorem{conv}{Convention}
\newtheorem{rem}{Remark}
\newtheorem{lem}{Lemma}
\newtheorem{cor}{Corollary}


\title{\bf{SPH4U: Complete Course Notes}}
\author{Gregory Olegovich Polstvin}
\date{Academic Year 2024-2025}

\begin{document}

\maketitle
\tableofcontents

\vspace{.25in}

\section{Unit 1: Kinematics}

\subsection{Terms and Definitions}

\begin{itemize}
  \item {\bf Scalars} A measured quantity with magnitude only. Examples
of this are: Time, distance, speed, mass, temperature, height, length,
volume, density, luminosity, work, and energy.

  \item {\bf Vectors} A measured quantity with both direction and magnitude.
Examples of this are: Force, acceleration, velocity, displacement, 
position, and momentum.

\item {\bf Displacement} Distance of the length of path traveled.
{\bf NOT} to be confused with distance, they are seperate concepts.

\item {\bf Instantaneous Velocity/Speed} The variable at one
moment in time.

\item {\bf Acceleration} The rate of change of velocity/change in velocity
per unit of time.

\end{itemize}

\subsection{Velocity}

{\bf Average Velocity} The total displacement over a given time interval.
Represented as:

\[\vec{v}{\text{avg}}=\frac{\Delta \vec{d}}{\Delta \vec{t}}

\subsection{Acceleration}

    {\bf Average Acceleration} The total change of velocity
over a given time interval. Represented as follows:

\[\vec{a}{\text{avg}}=\frac{\Delta \vec{V2} - \vec{V1}}{\Delta \vec{t}}\]

    {\bf Uniform Acceleration} The average acceleration for a time interval
equal to the instantaneous acceleration for ALL points
within that interval. \textbf{Acceleration
does not change.}

\end{document}
