\documentclass{article}

\usepackage{fancyhdr}
\usepackage{extramarks}
\usepackage{amsmath}
\usepackage{amsthm}
\usepackage{amsfonts}
\usepackage{siunitx}
\usepackage{textcomp}
\usepackage{tikz}


\usetikzlibrary{automata,positioning}

% Basic Document Settings

\topmargin=-0.45in
\evensidemargin=0in
\oddsidemargin=0in
\textwidth=6.5in
\textheight=9.0in
\headsep=0.25in

\linespread{1.1}

\pagestyle{fancy}
\lhead{\hmwkAuthorName}
\chead{\hmwkClass\ \hmwkClassInstructor}
\rhead{\firstxmark}
\lfoot{\lastxmark}
\cfoot{\thepage}

\renewcommand\headrulewidth{0.4pt}
\renewcommand\footrulewidth{0.4pt}

\setlength\parindent{0pt}

% Create Problem Sections

\newcommand{\enterProblemHeader}[1]{
    \nobreak\extramarks{}{Problem \arabic{#1} continued on next page\ldots}\nobreak{}
    \nobreak\extramarks{Problem \arabic{#1} (continued)}{Problem \arabic{#1} continued on next page\ldots}\nobreak{}
}

\newcommand{\exitProblemHeader}[1]{
    \nobreak\extramarks{Problem \arabic{#1} (continued)}{Problem \arabic{#1} continued on next page\ldots}\nobreak{}
    \stepcounter{#1}
    \nobreak\extramarks{Problem \arabic{#1}}{}\nobreak{}
}

\setcounter{secnumdepth}{0}
\newcounter{partCounter}
\newcounter{homeworkProblemCounter}
\setcounter{homeworkProblemCounter}{1}
\nobreak\extramarks{Problem \arabic{homeworkProblemCounter}}{}\nobreak{}

%
% Homework Problem Environment
%
% This environment takes an optional argument. When given, it will adjust the
% problem counter. This is useful for when the problems given for your
% assignment aren't sequential. See the last 3 problems of this template for an
% example.
%
\newenvironment{homeworkProblem}[1][-1]{
    \ifnum#1>0
        \setcounter{homeworkProblemCounter}{#1}
    \fi
    \section{Problem \arabic{homeworkProblemCounter}}
    \setcounter{partCounter}{1}
    \enterProblemHeader{homeworkProblemCounter}
}{
    \exitProblemHeader{homeworkProblemCounter}
}

%
% Homework Details
%   - Title
%   - Due date
%   - Class
%   - Section/Time
%   - Instructor
%   - Author
%

\newcommand{\hmwkTitle}{Asgnmt. 5}
\newcommand{\hmwkDueDate}{August 3, 2024}
\newcommand{\hmwkClass}{SPH4U}
\newcommand{\hmwkClassTime}{Unit 1}
\newcommand{\hmwkClassInstructor}{Naitrim Persaud}
\newcommand{\hmwkAuthorName}{\textbf{Gregory Olegovich Polstvin}}

% Title Page

\title{
    \vspace{2in}
    \textmd{\textbf{\hmwkClass:\ \hmwkTitle}}\\
    \normalsize\vspace{0.1in}\small{Due\ on\ \hmwkDueDate}\\
    \vspace{0.1in}\large{\textit{\hmwkClassInstructor \hmwkClassTime}}
    \vspace{3in}
}

\author{\hmwkAuthorName}
\date{}

\renewcommand{\part}[1]{\textbf{\large Part \Alph{partCounter}}\stepcounter{partCounter}\\}

% Alias for the Solution section header
\newcommand{\solution}{\textbf{\large Solution}}

% Probability cmds 
\newcommand{\E}{\mathrm{E}}
\newcommand{\Var}{\mathrm{Var}}
\newcommand{\Cov}{\mathrm{Cov}}
\newcommand{\Bias}{\mathrm{Bias}}

\begin{document}

\maketitle

\pagebreak

\begin{homeworkProblem}

A hiker used a compass to orient himself as he walked
\SI{3.00}{\km} in a direction \SI{50.0}{\degree} north of west,
and then an additional \SI{2.00}{\km} in a direction \SI{20.0}{\degree}
south of west. The trip took \SI{1.25}{\hour}. Calculate
the magnitude and direction of the hiker's velocity
of the hiker using vector components. \\

\textbf{Solution} \\

The magnitude and direction of the hiker's velocity is \SI{3.31}{\km\hour}
and \SI{23}{\degree} respectively.
\\


    \textbf{Sum of the displacement in the $x$ axis}
     \[
        x = \SI{3.00}{\km} \cdot \cos\SI{50}{\degree} -
        \SI{2.00}{\km} \cdot \cos\SI{20}{\degree}
     \]
     \[
     = \SI{-3.81}{\km}
     \] 
     \\
    \textbf{Sum of the displacement in the $y$ axis}
    \[
          y = \SI{3.00}{\km} \cdot \sin\SI{50}{\degree} -
          \SI{2.00}{\km} \cdot \sin\SI{20}{\degree}
    \]
    \[
            = \SI{1.614}{\km}
    \]
    \\

    \textbf{Magnitude of the resulting displacement}
    \[
        D = \sqrt{(y^2 + x^2)}
    \]
    \[
        = \sqrt{(\SI{-3.81}{\km})^2 + (\SI{1.61}{\km})^2} 
    \]
  \[
        = \sqrt{14.5 + 2.59}
      \]
    \[
        = \SI{4.13}{\km}
        \]
    \\

    \textbf{Velocity of the hiker}
    \]
    \\
    \textbf{Direction of the displacement}
    \[
    \]
    \\
\textbf{
    Thus, the magnitude and direction of the hiker's velocity is 3.31 km/h and  23⁰ respectively.

What is the magnitude of the hikers velocity?

The resultant displacement of the hiker is calculated as follows;

The sum of the displacement in y - direction;

Y = 3 km x sin(50) - 2 km x sin(20)

Y = 1.614 km

The sum of the displacement in x - direction;

X = -3 km x cos(50) - 2 km x cos(20)

X = - 3.81 km

The magnitude of the resultant displacement;

D = √ (Y² + X²)

D = √ ( 1.614² + 3.81²)

D = 4.14 km

The velocity of the hiker is calculated as follows;

v = D/t

v = 4.14 km / 1.25 hrs

v = 3.31 km/h

The direction of the displacement;

tan θ = Y/X

θ = arc tan (Y/X)

θ = arc tan (1.614 / 3.81)

θ = 23⁰

\end{homeworkProblem}

\end{document}
